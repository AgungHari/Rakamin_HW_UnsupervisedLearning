
\documentclass[12pt]{article}
\usepackage[utf8]{inputenc}
\usepackage{geometry}
\usepackage{graphicx}
\usepackage{amsmath}
\usepackage{threeparttable}
\usepackage{hyperref}
\usepackage{lipsum}
\usepackage{multirow}
%tambahkan package listing untuk kode python dan tambahkan warna agar listing lebih menarik
\usepackage{listings}
\usepackage{float}

\usepackage{xcolor}
\lstset{
    language=Python,
    basicstyle=\ttfamily\small,
    keywordstyle=\color{blue},
    commentstyle=\color{magenta},
    stringstyle=\color{red},
    numbers=left,
    numberstyle=\tiny\color{gray},
    stepnumber=1,
    numbersep=5pt,
    tabsize=4
}
\usepackage{fancyhdr}
\renewcommand{\figurename}{Gambar}
\geometry{margin=1in}

\title{Homework Solution - Unsupervised Learning}
\author{I Gusti Ngurah Agung Hari Vijaya Kusuma Batch 57}
\date{\today}

\begin{document}

\maketitle

\section*{Submission Links}
\begin{itemize}
  \item Repository: \href{https://github.com/AgungHari/Rakamin_HW_UnsupervisedLearning}{github.com/AgungHari}
\end{itemize}

\section{Pendahuluan}
\subsection{Latar Belakang}

Diberikan tugas untuk membuat model unsupervised learning dengan menggunakan dataset yang berisi data customer sebuah perusahaan penerbangan. Dataset ini mencakup berbagai fitur yang dapat menggambarkan nilai dari setiap customer, seperti ID Member, tanggal bergabung dalam program Frequent Flyer, jenis kelamin, tier program, kota asal, provinsi asal, negara asal, umur, jumlah penerbangan yang telah dilakukan, dan informasi terkait jarak penerbangan serta poin yang diperoleh.

Tujuan dari tugas ini adalah untuk menjawab Soal soal yang diberikan oleh Rakamin Academy. Dimana diharapkan outputnya dapat memberikan wawasan yang lebih dalam mengenai pola perilaku customer, segmentasi pasar, dan rekomendasi bisnis yang relevan berdasarkan hasil clustering. Dengan demikian, perusahaan penerbangan dapat mengoptimalkan strategi pemasaran dan meningkatkan pengalaman pelanggan.

\subsection{Homework Unsupervised Learning}

Adapun beberapa soal yang diberikan dalam tugas ini adalah sebagai berikut:
\begin{enumerate}
    \item Lakukan EDA pada dataset untuk mendapatkan pemahaman umum mengenai data dan memandu proses feature engineering (20 poin)
    \begin{itemize}
        \item Pastikan setiap kolom dataset memiliki tipe data yang tepat, tidak ada data kosong, bebas dari duplikat, dan berada di range value yang tepat.
        \item Keluarkan statistik kolom baik numerik maupun kategorikal, cari bentuk distribusi setiap kolom (numerik), dan jumlah baris untuk setiap unique value (kategorikal).
        \item Cari tahu apakah ada kolom-kolom yang berkorelasi kuat satu sama lain.
    \end{itemize}
    
    \item Pilih fitur-fitur yang menurut teman-teman masuk akal secara bisnis untuk digunakan sebagai fitur clustering. Lakukan feature engineering! (20 poin)
    \begin{itemize}
        \item Dari sekian banyak kolom yang ada, tentukan 3-6 fitur untuk digunakan sebagai fitur clustering. Tulis alasan teman-teman memilih fitur tersebut.
        \item Lakukan preprocessing dan feature engineering (apabila fitur yang teman-teman pilih merupakan fitur baru yang dihasilkan dari fitur-fitur yang sudah ada).
    \end{itemize}
    
    \item Lakukan clustering K-means! Temukan jumlah cluster yang menurut teman-teman optimal dan evaluasi cluster yang dihasilkan dengan visualisasi dan silhouette score (30 poin)
    \begin{itemize}
        \item Temukan jumlah cluster yang optimal dengan menggunakan elbow method.
        \item Lakukan clustering menggunakan K-means.
        \item Evaluasi cluster yang dihasilkan dengan menggunakan visualisasi, gunakan PCA apabila diperlukan.
    \end{itemize}
    
    \item Interpretasi cluster yang dihasilkan secara bisnis dan berikan rekomendasi yang sesuai dengan cluster yang dihasilkan (30 poin)
    \begin{itemize}
        \item Tempelkan kembali label yang dihasilkan ke dataframe asal, dan keluarkan statistik fitur dari setiap cluster.
        \item Deskripsikan secara kontekstual customer seperti apa yang ada di masing-masing cluster.
        \item Berdasarkan cluster tersebut, berikan 1-2 rekomendasi bisnis.
    \end{itemize}
\end{enumerate}




\newpage
\section{Tinjauan Pustaka}

% berikut adalah fitur dari dataset buatkan tinjauan pusataka yang merepresentasikan fitur-fitur 
% Code Description
% - MEMBER_NO-b : ID Member
% - FFP_DATE : Frequent Flyer Program Join Date
% - FIRST_FLIGHT_DATE : Tanggal Penerbangan pertama
% - GENDER : Jenis Kelamin
% - FFP_TIER : Tier dari Frequent Flyer Program
% - WORK_CITY : Kota Asal
% - WORK_PROVINCE : Provinsi Asal
% - WORK_COUNTRY : Negara Asal
% - AGE : Umur Customer
% - LOAD_TIME : Tanggal data diambil
% - FLIGHT_COUNT : Jumlah penerbangan Customer
% - BP_SUM : Rencana Perjalanan
% - SUM_YR_1 : Fare Revenue
% - SUM_YR_2 : Votes Prices
% - SEG_KM_SUM : Total jarak(km) penerbangan yg sudah dilakukan
% - LAST_FLIGHT_DATE : Tanggal penerbangan terakhir
% - LAST_TO_END : Jarak waktu penerbangan terakhir ke pesanan penerbangan paling akhir
% - AVG_INTERVAL : Rata-rata jarak waktu
% - MAX_INTERVAL : Maksimal jarak waktu
% - EXCHANGE_COUNT : Jumlah penukaran
% - avg_discount : Rata rata discount yang didapat customer
% - Points_Sum : Jumlah poin yang didapat customer
% - Point_NotFlight : point yang tidak digunakan oleh members

\subsection{Frequent Flyer Program}
Frequent Flyer Program (FFP) adalah program loyalitas yang ditawarkan oleh maskapai penerbangan kepada pelanggan setia mereka. Program ini memberikan berbagai keuntungan, seperti akumulasi poin atau miles yang dapat ditukarkan dengan tiket penerbangan gratis, peningkatan kelas penerbangan, akses ke lounge bandara, dan layanan prioritas lainnya. FFP dirancang untuk mendorong pelanggan agar terus menggunakan layanan maskapai tertentu, sehingga meningkatkan retensi pelanggan dan loyalitas merek.

Program ini biasanya memiliki beberapa tingkatan atau tier, yang memberikan manfaat tambahan kepada anggota yang mencapai tingkat tertentu berdasarkan frekuensi penerbangan atau jumlah poin yang dikumpulkan. Dengan demikian, FFP tidak hanya memberikan insentif bagi pelanggan untuk terbang lebih sering, tetapi juga menciptakan hubungan jangka panjang antara maskapai dan pelanggannya.

\begin{figure}[H]
    \centering
    \includegraphics[width=0.6\textwidth]{gambar/ffp.jpg}
    \caption{Contoh Frequent Flyer Program}
    \label{fig:ffp}
\end{figure}

\subsection{EDA (Exploratory Data Analysis)}
Exploratory Data Analysis (EDA) adalah proses analisis data yang bertujuan untuk memahami struktur, pola, dan hubungan dalam dataset sebelum melakukan analisis lebih lanjut atau membangun model. EDA melibatkan penggunaan berbagai teknik statistik dan visualisasi untuk mengeksplorasi data, mengidentifikasi anomali, dan mendapatkan wawasan awal tentang karakteristik data.

Proses EDA biasanya mencakup langkah-langkah seperti:
\begin{itemize}
    \item \textbf{Pemeriksaan Data:} Memeriksa tipe data, nilai yang hilang, dan distribusi variabel.
    \item \textbf{Statistik Deskriptif:} Menghitung ukuran pusat (mean, median) dan ukuran dispersi (standar deviasi, rentang).
    \item \textbf{Visualisasi Data:} Menggunakan grafik seperti histogram, boxplot, dan scatter plot untuk memahami distribusi dan hubungan antar variabel.
    \item \textbf{Identifikasi Outlier:} Mendeteksi nilai-nilai yang tidak biasa yang dapat mempengaruhi analisis.
    \item \textbf{Korelasi:} Menganalisis hubungan antar variabel untuk mengidentifikasi pola yang mungkin ada.
\end{itemize}

EDA sangat penting dalam tahap awal analisis data karena membantu peneliti atau analis untuk memahami data secara mendalam, mengarahkan fokus pada area yang relevan, dan menginformasikan keputusan tentang teknik analisis yang akan digunakan selanjutnya. Dengan demikian, EDA merupakan langkah krusial dalam proses analisis data yang efektif.

\subsection{Data Preprocessing}
Data preprocessing adalah langkah penting dalam analisis data yang melibatkan pembersihan, transformasi, dan persiapan data sebelum digunakan dalam analisis atau pemodelan. Tujuan dari preprocessing adalah untuk memastikan bahwa data dalam kondisi yang baik, konsisten, dan siap untuk dianalisis. Langkah-langkah umum dalam data preprocessing meliputi:
\begin{itemize}
    \item \textbf{Pembersihan Data:} Menghapus atau memperbaiki data yang tidak lengkap, duplikat, atau tidak konsisten. Ini termasuk menangani nilai yang hilang, mengoreksi kesalahan pengetikan, dan menghapus outlier yang tidak relevan.
    \item \textbf{Transformasi Data:} Mengubah format data agar sesuai dengan kebutuhan analisis. Ini bisa meliputi normalisasi atau standarisasi nilai numerik, pengkodean variabel kategorikal, dan konversi tipe data.
    \item \textbf{Penggabungan Data:} Menggabungkan beberapa sumber data menjadi satu dataset yang kohesif, jika diperlukan.
    \item \textbf{Pemisahan Data:} Membagi dataset menjadi subset untuk pelatihan dan pengujian model, terutama dalam konteks machine learning.
    \item \textbf{Feature Engineering:} Membuat fitur baru dari data yang ada untuk meningkatkan kinerja model. Ini bisa meliputi penggabungan variabel, ekstraksi informasi dari teks, atau pembuatan variabel waktu.
    \item \textbf{Skalasi Data:} Mengubah skala fitur numerik agar berada dalam rentang yang sama, yang penting untuk algoritma yang sensitif terhadap skala, seperti K-Means atau SVM.
\end{itemize}

\subsection{K-Means Clustering}
K-Means Clustering adalah algoritma pembelajaran tidak terawasi yang digunakan untuk mengelompokkan data ke dalam sejumlah kluster berdasarkan kesamaan fitur. Algoritma ini bekerja dengan cara membagi dataset menjadi K kluster, di mana setiap kluster diwakili oleh centroid (titik pusat kluster). 
%rumus k-means clustering
Proses K-Means Clustering melibatkan langkah-langkah berikut:
\begin{enumerate}
    \item \textbf{Inisialisasi Centroid:} Memilih K titik acak dari dataset sebagai centroid awal.
    \item \textbf{Penugasan Kluster:} Menghitung jarak antara setiap titik data dan centroid, lalu mengelompokkan setiap titik ke kluster terdekat.
    \item \textbf{Pembaruan Centroid:} Menghitung ulang centroid untuk setiap kluster berdasarkan rata-rata posisi titik-titik dalam kluster tersebut.
    \item \textbf{Iterasi:} Mengulangi langkah 2 dan 3 hingga centroid tidak berubah secara signifikan atau jumlah iterasi maksimum tercapai.
    \item \textbf{Output:} Menghasilkan kluster yang berisi titik-titik data yang dikelompokkan berdasarkan kesamaan fitur.
\end{enumerate}

Adapun rumus untuk menghitung jarak antara titik data dan centroid biasanya menggunakan Euclidean distance, yang didefinisikan sebagai:

\begin{equation}
    d(x, c) = \sqrt{\sum_{i=1}^{n} (x_i - c_i)^2}
\end{equation}
Di mana :
\begin{itemize}
    \item \(d(x, c)\) adalah jarak antara titik data \(x\) dan centroid \(c\),
    \item \(x_i\) adalah nilai fitur ke-i dari titik data \(x\),
    \item \(c_i\) adalah nilai fitur ke-i dari centroid \(c\),
    \item \(n\) adalah jumlah fitur.
    \item \(x\) adalah titik data yang akan dikelompokkan.
    \item \(c\) adalah centroid dari kluster yang sedang dianalisis.
\end{itemize}

K-Means Clustering banyak digunakan dalam berbagai aplikasi, seperti segmentasi pasar, pengelompokan dokumen, dan analisis citra. Kelebihan dari algoritma ini adalah kesederhanaannya dan efisiensi dalam menangani dataset besar. Namun, K-Means juga memiliki beberapa kelemahan, seperti ketergantungan pada pemilihan jumlah kluster \(K\) yang tepat dan sensitivitas terhadap outlier.

\subsection{Elbow Method}
Elbow Method adalah teknik yang digunakan untuk menentukan jumlah optimal kluster \(K\) dalam algoritma K-Means Clustering. Metode ini melibatkan pengukuran varians dalam kluster (inertia) untuk berbagai nilai \(K\) dan kemudian memplot hasilnya. Tujuan dari Elbow Method adalah untuk menemukan titik di mana penambahan kluster baru tidak memberikan peningkatan signifikan dalam pengurangan varians.

Proses Elbow Method meliputi langkah-langkah berikut:
\begin{enumerate}
    \item \textbf{Inisialisasi K-Means:} Jalankan algoritma K-Means untuk berbagai nilai \(K\) (misalnya, dari 1 hingga 10).
    \item \textbf{Hitung Inertia:} Untuk setiap nilai \(K\), hitung inertia, yaitu jumlah jarak kuadrat antara titik data dan centroid kluster mereka.
    \item \textbf{Plot Inertia:} Buat plot dengan nilai \(K\) pada sumbu x dan inertia pada sumbu y.
    \item \textbf{Identifikasi Elbow:} Cari titik di mana penurunan inertia mulai melambat, yang biasanya terlihat seperti "siku" pada plot. Titik ini menunjukkan jumlah kluster optimal.
    \item \textbf{Pilih K Optimal:} Nilai \(K\) pada titik siku ini dianggap sebagai jumlah kluster yang paling sesuai untuk dataset.
\end{enumerate}

Elbow Method membantu dalam menghindari overfitting dengan memilih jumlah kluster yang tepat, sehingga model K-Means dapat menangkap struktur data dengan baik tanpa menjadi terlalu kompleks. Meskipun metode ini sederhana dan intuitif, hasilnya dapat bervariasi tergantung pada dataset dan distribusi data, sehingga penting untuk mempertimbangkan konteks analisis saat menentukan jumlah kluster optimal.

\subsection{Silhouette Score}
Silhouette Score adalah metrik yang digunakan untuk mengevaluasi kualitas klustering dalam algoritma K-Means atau metode klustering lainnya. Metrik ini mengukur seberapa baik setiap titik data dikelompokkan dalam kluster yang benar, dengan mempertimbangkan jarak antar titik dalam kluster dan jarak ke titik di kluster lain.
Silhouette Score untuk setiap titik data dihitung dengan rumus berikut:

\begin{equation}
    s(i) = \frac{b(i) - a(i)}{\max(a(i), b(i))}
\end{equation}

Di mana:
\begin{itemize}
    \item \(s(i)\) adalah Silhouette Score untuk titik data \(i\),
    \item \(a(i)\) adalah rata-rata jarak antara titik data \(i\) dan semua titik lain dalam kluster yang sama (intra-kluster),
    \item \(b(i)\) adalah rata-rata jarak antara titik data \(i\) dan titik-titik di kluster terdekat lainnya (inter-kluster).
    \item \(i\) adalah indeks dari titik data yang sedang dianalisis.
\end{itemize}

Nilai Silhouette Score berkisar antara -1 hingga 1:
\begin{itemize}
    \item Nilai mendekati 1 menunjukkan bahwa titik data berada jauh dari kluster lain dan dekat dengan kluster yang benar.
    \item Nilai mendekati 0 menunjukkan bahwa titik data berada di batas antara dua kluster.
    \item Nilai negatif menunjukkan bahwa titik data mungkin telah dikelompokkan ke kluster yang salah.
\end{itemize}

Silhouette Score memberikan gambaran tentang seberapa baik klustering dilakukan, dengan nilai yang lebih tinggi menunjukkan klustering yang lebih baik. Metrik ini berguna untuk membandingkan hasil klustering dengan jumlah kluster yang berbeda dan membantu dalam memilih jumlah kluster yang optimal.

\subsection{PCA (Principal Component Analysis)}
Principal Component Analysis (PCA) adalah teknik reduksi dimensi yang digunakan untuk mengurangi jumlah variabel dalam dataset sambil mempertahankan sebanyak mungkin informasi yang ada. PCA bekerja dengan mengidentifikasi arah (komponen utama) di mana data memiliki varians terbesar, sehingga memungkinkan representasi data dalam ruang dimensi yang lebih rendah.

Proses PCA melibatkan langkah-langkah berikut:
\begin{enumerate}
    \item \textbf{Standardisasi Data:} Mengubah data sehingga memiliki rata-rata 0 dan deviasi standar 1 untuk setiap fitur, agar setiap fitur berkontribusi secara setara.
    \item \textbf{Kovarians Matriks:} Menghitung matriks kovarians untuk memahami hubungan antar fitur dalam dataset.
    \item \textbf{Eigen Decomposition:} Menghitung nilai eigen (eigenvalues) dan vektor eigen (eigenvectors) dari matriks kovarians. Vektor eigen menunjukkan arah komponen utama, sedangkan nilai eigen menunjukkan seberapa banyak varians yang dijelaskan oleh masing-masing komponen.
    \item \textbf{Pemilihan Komponen Utama:} Memilih sejumlah komponen utama berdasarkan nilai eigen terbesar, yang akan digunakan untuk merepresentasikan data dalam dimensi yang lebih rendah.
    \item \textbf{Transformasi Data:} Mengalikan data asli dengan vektor eigen terpilih untuk mendapatkan representasi baru dalam ruang dimensi yang lebih rendah.
\end{enumerate}

PCA sangat berguna dalam mengurangi kompleksitas data, menghilangkan redundansi, dan meningkatkan efisiensi komputasi dalam analisis data. Selain itu, PCA juga membantu dalam visualisasi data dengan mengurangi dimensi menjadi 2 atau 3 komponen utama, sehingga memudahkan pemahaman pola dan struktur dalam dataset. Namun, penting untuk diingat bahwa PCA adalah teknik linier, sehingga mungkin tidak cocok untuk semua jenis data, terutama yang memiliki hubungan non-linier yang kompleks.

\subsection{RFM (Recency, Frequency, Monetary)}
RFM (Recency, Frequency, Monetary) adalah metode analisis yang digunakan untuk mengukur nilai pelanggan berdasarkan tiga dimensi utama:

\begin{itemize}
    \item \textbf{Recency (R):} Mengukur seberapa baru pelanggan melakukan pembelian. Semakin baru pembelian, semakin tinggi nilai recency.
    \item \textbf{Frequency (F):} Mengukur seberapa sering pelanggan melakukan pembelian dalam periode tertentu. Semakin sering pembelian, semakin tinggi nilai frequency.
    \item \textbf{Monetary (M):} Mengukur total pengeluaran pelanggan dalam periode tertentu. Semakin besar pengeluaran, semakin tinggi nilai monetary.
\end{itemize}






\newpage
\section{Desain dan Implementasi}

\subsection{Deskripsi sistem}

\newpage

\section{Pengujian dan analisis}
% bab 4 akan menjelaskan hasil train dari model clustering k means yang telah dibuat
Pada bab ini, akan dijelaskan mengenai hasil pengujian dan analisis dari model clustering yang menggunakan algoritma K-Means. Selain itu, akan dipaparkan juga mengenai skenario pengujian yang dilakukan untuk mengevaluasi performa sistem dalam mengelompokkan data pelanggan berdasarkan fitur-fitur yang relevan. Pengujian ini bertujuan untuk memastikan bahwa model K-Means yang dirancang dapat mengidentifikasi pola dalam data dan memberikan hasil yang akurat dalam segmentasi pelanggan.

\subsection{Pengujian Sistem}

Pengujian sistem dilakukan dengan menggunakan dataset yang telah diolah sebelumnya. Data tersebut akan digunakan untuk melatih model clustering K-Means yang diharapkan dapat menghasilkan cluster yang merepresentasikan segmen pelanggan berdasarkan fitur fitur yang dipilih pada bab sebelumnya. Adapun skenario pengujian yang dilakukan adalah sebagai berikut:

\begin{enumerate}
    \item \texttt{Pengujian KMeans} : Data yang telah diolah sebelumnya akan digunakan dalam pelatihan KMeans untuk mendapatkan model clustering yang optimal.
    \item \texttt{Evaluasi Hasil Clustering} : Setelah model KMeans dilatih, hasil clustering akan dievaluasi untuk menentukan seberapa baik model tersebut dalam mengelompokkan data pelanggan.
\end{enumerate}

\subsection{Pengujian KMeans}

Pengujian KMeans dilakukan dengan menggunakan data yang telah dipersiapkan sebelumnya. Data tersebut terdiri dari fitur-fitur yang relevan untuk segmentasi pelanggan, seperti \texttt{SUM\_YR\_1}, \texttt{LAST\_TO\_END}, dan \texttt{AVG\_INTERVAL}. Model KMeans akan dilatih dengan data ini untuk mengidentifikasi pola dan mengelompokkan pelanggan ke dalam cluster yang sesuai.

\subsubsection{Hasil Pengujian KMeans dengan Elbow Method}
Sebelum kita masuk ke hasil clustering, kita perlu menentukan jumlah cluster yang optimal. Salah satu metode yang umum digunakan untuk menentukan jumlah cluster adalah Elbow Method. Metode ini melibatkan perhitungan inertia (dalam hal ini, jarak kuadrat dari titik data ke pusat cluster) untuk berbagai jumlah cluster dan kemudian memplot hasilnya. Gambar \ref{fig:elbow_method} menunjukkan plot Elbow Method yang digunakan untuk menentukan jumlah cluster yang optimal.

\begin{figure}[H]
    \centering
    \includegraphics[width=0.5\textwidth]{gambar/ElbowMethod.png}
    \caption{Plot Elbow Method untuk menentukan jumlah cluster optimal}
    \label{fig:elbow_method}
\end{figure}

Didapatkan jumlah cluster yang optimal adalah 4, karena pada titik tersebut terjadi penurunan inertia yang signifikan sehingga cluster tersebut akan digunakan untuk visualisasi hasil clustering KMeans selanjutnya.

\subsubsection{Hasil Clustering KMeans}
Setelah menentukan jumlah cluster yang optimal, model KMeans dilatih dengan data yang telah disiapkan. Hasil clustering menunjukkan bahwa pelanggan dikelompokkan ke dalam 4 cluster yang berbeda. Gambar \ref{fig:kmeans_clusters} menunjukkan hasil clustering 3D Plot untuk KMeans yang telah dilakukan.

\begin{figure}[H]
    \centering
    \includegraphics[width=0.65\textwidth]{gambar/hasilkmeans2.png}
    \caption{Hasil Clustering KMeans 3D Plot}
    \label{fig:kmeans_clusters}
\end{figure}

Dari plot 3D di atas, kita dapat melihat bahwa pelanggan dikelompokkan ke dalam 4 cluster yang berbeda berdasarkan fitur-fitur yang telah dipilih. Setiap titik mewakili pelanggan, dan warna yang berbeda menunjukkan cluster yang berbeda. Hal ini menunjukkan bahwa model KMeans berhasil mengidentifikasi pola dalam data dan mengelompokkan pelanggan dengan cara yang bermakna.

Adapun warna yang digunakan untuk setiap segmen adalah sebagai berikut:
\begin{itemize}
    \item Segmen VIP : Hijau
    \item Segmen Customer Active Frequent Low Spender : Kuning
    \item Segmen Lapsed Frequent Low Spender : Biru
    \item Segmen Occasional Low Value : Merah
\end{itemize}

Dari hasil tersebut kita telah mendapatkan 4 segmen pelanggan yang berbeda, yang masing-masing memiliki karakteristik dan perilaku yang unik. Hal ini memungkinkan perusahaan untuk merancang strategi pemasaran yang lebih efektif dan terfokus pada setiap segmen pelanggan.

Selain menggunakan 3D plot kita juga akan menggunakan PCA untuk memvisualisasikan hasil clustering KMeans. Gambar \ref{fig:pca_kmeans} menunjukkan hasil PCA yang digunakan untuk mengurangi dimensi data dan memvisualisasikan cluster dalam 2D.

\begin{figure}[H]
    \centering
    \includegraphics[width=0.65\textwidth]{gambar/PCA.png}
    \caption{Hasil PCA Clustering KMeans 2D Plot}
    \label{fig:pca_kmeans}
\end{figure}

Dari plot PCA di atas, kita dapat melihat bahwa pelanggan dikelompokkan ke dalam 4 cluster yang berbeda dengan jelas. Setiap titik mewakili pelanggan, dan warna yang berbeda menunjukkan cluster yang berbeda. Hal ini menunjukkan bahwa model KMeans berhasil mengidentifikasi pola dalam data dan mengelompokkan pelanggan dengan cara yang bermakna.

Selain dari plot didapatkan juga ukuran pemusatan dari hasil Clustering KMeans yang telah dilakukan. Tabel \ref{tab:cluster_summary} menunjukkan pusat dari setiap cluster yang dihasilkan oleh model KMeans.

% SUM_YR_1	LAST_TO_END	AVG_INTERVAL
% clusters	mean	    median	mean	median	mean	median					
% 0	        2027.909230	1709.0	149.954424	123.0	111.947414	107.000000
% 1	        3189.024061	2639.5	462.502805	463.0	31.555278	27.775000
% 2	        9795.947353	9294.0	79.781578	42.0	35.590739	32.547727
% 3	        2035.480580	1700.0	89.106335	69.0	42.410628	43.600000

\begin{table}[H]
    \centering
    \caption{Statistik Rata-rata dan Median per Cluster}
    \label{tab:cluster_summary}
    \begin{tabular}{|c|c|c|c|c|c|c|}
        \hline
        \multirow{2}{*}{Cluster} & \multicolumn{2}{c|}{SUM\_YR\_1} & \multicolumn{2}{c|}{LAST\_TO\_END} & \multicolumn{2}{c|}{AVG\_INTERVAL} \\
        \cline{2-7}
         & Mean & Median & Mean & Median & Mean & Median \\
        \hline
        0 & 2027.91 & 1709.0 & 149.95 & 123.0 & 111.95 & 107.0 \\
        1 & 3189.02 & 2639.5 & 462.50 & 463.0 & 31.56  & 27.78 \\
        2 & 9795.95 & 9294.0 & 79.78  & 42.0  & 35.59  & 32.55 \\
        3 & 2035.48 & 1700.0 & 89.11  & 69.0  & 42.41  & 43.60 \\
        \hline
    \end{tabular}
\end{table}

Tabel di atas menunjukkan rata-rata dan median dari setiap fitur untuk masing-masing cluster. Dari tabel tersebut, kita dapat melihat perbedaan yang signifikan antara cluster yang satu dengan yang lainnya. Hal ini menunjukkan bahwa model KMeans berhasil mengidentifikasi pola dalam data dan mengelompokkan pelanggan dengan cara yang bermakna.

\subsection{Interpretasi Hasil Clustering}
% \begin{enumerate}
%     \item Recency Rendah, Frequency Rendah, Monetary Tinggi merupakan Customer VIP.
%     \item Recency Rendah, Frequency Rendah, Monetary Rendah merupakan Customer Active Frequent Low Spender.
%     \item Recency Tinggi, Frequency Rendah, Monetary Rendah merupakan Lapsed Frequent Low Spender.
%     \item Recency Rendah, Frequency Tinggi, Monetary Rendah merupakan Occasional Low Value.
% \end{enumerate}
Interpretasi hasil clustering KMeans memberikan wawasan yang berharga tentang segmen pelanggan yang berbeda. Berdasarkan hasil clustering, kita dapat mengidentifikasi empat segmen pelanggan yang berbeda, masing-masing dengan karakteristik dan perilaku yang unik. Namun sebelum itu Tabel \ref{tab:cluster_summary} di atas memberikan tabel makna dari masing masing RFM:

\begin{table}[H]
    \centering
    \caption{Makna dari masing-masing RFM}
    \label{tab:rfm_meaning}
    \begin{tabular}{|c|c|}
        \hline
        RFM & Makna \\
        \hline
        Recency Rendah, Frequency Rendah, Monetary Tinggi & Customer VIP \\
        Recency Rendah, Frequency Rendah, Monetary Rendah & Customer Active Frequent Low Spender \\
        Recency Tinggi, Frequency Rendah, Monetary Rendah & Lapsed Frequent Low Spender \\
        Recency Rendah, Frequency Tinggi, Monetary Rendah & Occasional Low Value \\
        \hline
    \end{tabular}
\end{table}

Di mana :
\begin{itemize}
    \item Semakin rendah Recency berarti pelanggan tersebut semakin baru melakukan transaksi, berdasarkan makna dari fitur LAST\_TO\_END.
    \item Semakin rendah Frequency berarti pelanggan tersebut semakin sering melakukan transaksi, berdasarkan makna dari fitur AVG\_INTERVAL.
    \item Semakin tinggi Monetary berarti pelanggan tersebut semakin mahal transaksinya, berdasarkan makna dari fitur SUM\_YR\_1.
\end{itemize}

Sehingga bisa kita lakukan Interpretasi dari 4 cluster yang didapatkan berdasarkan warna yang ada pada hasil clustering KMeans:


\begin{enumerate}
    \item \textbf{Segmen VIP (Hijau)}: Pelanggan dengan Recency rendah, Frequency rendah, dan Monetary tinggi. Mereka adalah pelanggan yang jarang bertransaksi tetapi ketika bertransaksi, mereka mengeluarkan uang dalam jumlah besar. Ini menunjukkan bahwa mereka adalah pelanggan yang sangat berharga bagi perusahaan.
    \item \textbf{Segmen Customer Active Frequent Low Spender (Kuning)}: Pelanggan dengan Recency rendah, Frequency rendah, dan Monetary rendah. Mereka adalah pelanggan yang sering bertransaksi tetapi dengan nilai transaksi yang rendah. Ini menunjukkan bahwa mereka adalah pelanggan yang aktif tetapi tidak mengeluarkan banyak uang.
    \item \textbf{Segmen Lapsed Frequent Low Spender (Biru)}: Pelanggan dengan Recency tinggi, Frequency rendah, dan Monetary rendah. Mereka adalah pelanggan yang jarang bertransaksi dan ketika bertransaksi, mereka mengeluarkan uang dalam jumlah kecil. Ini menunjukkan bahwa mereka mungkin telah kehilangan minat pada produk atau layanan perusahaan.
    \item \textbf{Segmen Occasional Low Value (Merah)}: Pelanggan dengan Recency rendah, Frequency tinggi, dan Monetary rendah. Mereka adalah pelanggan yang sering bertransaksi tetapi dengan nilai transaksi yang sangat rendah. Ini menunjukkan bahwa mereka mungkin hanya membeli produk atau layanan dengan harga murah.
\end{enumerate}

Dengan memahami karakteristik masing-masing segmen pelanggan, perusahaan dapat merancang strategi pemasaran yang lebih efektif dan terfokus pada setiap segmen. Misalnya, perusahaan dapat menawarkan promosi khusus kepada pelanggan VIP untuk meningkatkan loyalitas mereka, atau menawarkan diskon kepada pelanggan Lapsed Frequent Low Spender untuk menarik mereka kembali bertransaksi.

\newpage
\section{Kesimpulan dan Saran}

Dari hasil pengujian dan analisis yang telah dilakukan, dapat disimpulkan bahwa model KMeans berhasil mengelompokkan pelanggan ke dalam 4 segmen yang berbeda berdasarkan fitur-fitur yang relevan. Adapun poin poin kesimpulan yabng dapat diambil dari bab ini adalah sebagai berikut:

\begin{enumerate}
    \item Model KMeans berhasil mengidentifikasi pola dalam data dan mengelompokkan pelanggan dengan cara yang bermakna.
    \item Penggunaan Elbow Method membantu dalam menentukan jumlah cluster yang optimal, yaitu 4 cluster.
    \item Hasil clustering menunjukkan perbedaan yang signifikan antara segmen pelanggan, yang memungkinkan perusahaan untuk merancang strategi pemasaran yang lebih efektif.
    \item Interpretasi hasil clustering memberikan wawasan yang berharga tentang karakteristik dan perilaku pelanggan, yang dapat digunakan untuk meningkatkan pengalaman pelanggan dan loyalitas mereka.
\end{enumerate}

Adapun saran yang dapat diberikan untuk penelitian selanjutnya adalah sebagai berikut:

\begin{enumerate}
    \item Melakukan eksperimen dengan algoritma clustering lain seperti DBSCAN atau Hierarchical Clustering untuk membandingkan hasilnya dengan KMeans.
    \item Menerapkan teknik feature engineering yang lebih kompleks untuk meningkatkan kualitas data dan hasil clustering, seperti normalisasi atau transformasi logaritmik pada fitur numerik.
    \item Menggunakan teknik visualisasi yang lebih interaktif dan mendalam untuk memahami pola dalam data, seperti menggunakan t-SNE atau PCA untuk mereduksi dimensi sebelum clustering.
\end{enumerate}

\end{document}