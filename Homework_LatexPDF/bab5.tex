\newpage
\section{Kesimpulan dan Saran}

Dari hasil pengujian dan analisis yang telah dilakukan, dapat disimpulkan bahwa model KMeans berhasil mengelompokkan pelanggan ke dalam 4 segmen yang berbeda berdasarkan fitur-fitur yang relevan. Adapun poin poin kesimpulan yabng dapat diambil dari bab ini adalah sebagai berikut:

\begin{enumerate}
    \item Model KMeans berhasil mengidentifikasi pola dalam data dan mengelompokkan pelanggan dengan cara yang bermakna.
    \item Penggunaan Elbow Method membantu dalam menentukan jumlah cluster yang optimal, yaitu 4 cluster.
    \item Hasil clustering menunjukkan perbedaan yang signifikan antara segmen pelanggan, yang memungkinkan perusahaan untuk merancang strategi pemasaran yang lebih efektif.
    \item Interpretasi hasil clustering memberikan wawasan yang berharga tentang karakteristik dan perilaku pelanggan, yang dapat digunakan untuk meningkatkan pengalaman pelanggan dan loyalitas mereka.
\end{enumerate}

Adapun saran yang dapat diberikan untuk penelitian selanjutnya adalah sebagai berikut:

\begin{enumerate}
    \item Melakukan eksperimen dengan algoritma clustering lain seperti DBSCAN atau Hierarchical Clustering untuk membandingkan hasilnya dengan KMeans.
    \item Menerapkan teknik feature engineering yang lebih kompleks untuk meningkatkan kualitas data dan hasil clustering, seperti normalisasi atau transformasi logaritmik pada fitur numerik.
    \item Menggunakan teknik visualisasi yang lebih interaktif dan mendalam untuk memahami pola dalam data, seperti menggunakan t-SNE atau PCA untuk mereduksi dimensi sebelum clustering.
\end{enumerate}